% !TeX root = Main.tex

V~posledních desetiletích se při vývoji nových zařízení nehledí jen na~to, aby daný výrobek plnil svojí primární funkci, pro kterou byl navržen, ale předmětem zájmu jsou také ostatní provozní vlastnosti. Mezi nejvýznamnější se řadí činnost bez vytváření nepřípustné úrovně elektromagnetického rušení a bezchybné fungování v~daném prostředí. Vědecká disciplína, která se danou problematikou podrobněji zabývá je označována pojmem elektromagnetická kompatibilita (z~anglického \uv{Electromagnetic Compatibility}, označované zkratkou EMC). Významnost tohoto specifického oborou narostla i díky rozvoji a rozšíření elektrotechniky, zvláště pak číslicové mikropočítačové elektroniky, která je dnes využita v řadě technických oblastí.

Existuje celá řada aplikací, ve kterých se šíření elektromagnetického pole může negativně projevovat. To se týká poměrně jednoduchých zapojení, složitějších analogových nebo digitálních obvodů, až komplexních procesorových systémů. EMC je v~této práci zaměřena na obor elektrické trakce, neboť v~tomto segmentu dochází k~častému prolínání řady systémů různých výkonů, topologií i principů činnosti. Nutno podotknout, že vliv elektromagnetických vln nemusí být vždy negativní. V~řadě aplikací představují vůbec základní element fungování. Typickým příkladem je oblast komunikací. Jedná o~využití radiových vln v~radiotechnice nebo při konkrétní konstrukci používaných zařízení, např. vlnovodů, rezonátorů.

Problematika elektromagnetické kompatibility zařízení velmi široce souvisí s~návrhem a konstrukcí. Především již ve fázi návrhu zařízení se uvažuje nad tím, v~jakém prostředí bude používané. V~řadě aplikací ale nelze z~provozních hledisek provádět funkční zkoušky prototypů v~konkrétním prostředí. Nejen z~tohoto důvodu narůstá význam simulačních aplikací, díky kterým je možné předem nasimulovat např. schopnost použitého stínění nebo účinnost navržené antény. Mezi další přínosy může přirozeně patřit urychlení vývoje nebo snížení nákladů na výrobu. Právě tvorba aplikace pro možnost simulace elektromagnetického pole je ústředním tématem této diplomové práce. Konkrétněji o~tom pojednává kapitola \ref{kap:Cile}.

Práce je složena z~několika kapitol. Kapitola \ref{kap:Cile} vytyčuje konkrétní cíle, které tato práce hodlá splnit, kapitola \ref{kap:EMC} se s pomocí \cite{emc_trakce}, \cite{nfr}, \cite{emc_encyklopedie} a \cite{csn} zabývá problematikou EMC v~konkrétním technickém oboru, v~elektrické trakci. V~další kapitole \ref{kap:Evlny} je dle \cite{emp}, \cite{umt} a \cite{tripak} rozebrán matematický popis elektromagnetických vln a popsáno jejich chování v~různých fyzikálních prostředích. Návrhem aplikace pro simulace šíření elektromagnetických vln se s využitím \cite{num}, \cite{hpfem}, \cite{gk_kaw} a \cite{gk_tichy} zabývá kapitola \ref{kap:Simulace}, ilustrativní příklad řešení je popsán v kapitole \ref{kap:Priklad}. V~závěrečné kapitole \ref{kap:Zaver} jsou shrnuty dosažené poznatky a popsány výsledky práce. V~příloze \ref{kap:Odvozeni_VlnR}~se nachází podrobné odvození vlnových rovnic elektromagnetického pole, v příloze \ref{kap:tutorial} je popis řešení úloh v aplikaci Agros2D. Příloha \ref{kap:Program_kod} je shrnutím nejdůležitější částí programového kódu simulačního modulu.

