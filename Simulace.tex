% !TeX root = Main.tex
Programový kód, kterým je realizována simulace šíření elektromagnetického pole, je proveden jako modul pro aplikaci Agros2D. Využívá proto její preprocesor pro zadávání dat a také postprocesor pro zobrazení výsledků ve 2D a 3D zobrazení. Vlastní numerické řešení zajišťuje knihovna Hermes2D. V této kapitole jsou popsány výpočetní části programu daného modulu, které se týkají především řešení vlnových rovnic a specifikace okrajových podmínek. 

\section{Předpoklady simulace}
Vzhledem k široké problematice elektromagnetických vln je potřeba pro jejich modelování zavést některé zjednodušení. 
\begin{itemize}
\item {\bf Harmonická analýza} - první předpokladem je řešení vlnové rovnice v harmonickém tvaru (Helmholtzova rovnice)
\begin{displaymath}
	\nabla^{2}\vecfaz E +\faz k^{2}\vecfaz E = 0.
\end{displaymath}
\item {\bf Šíření vln v kartézské souřadnicové soustavě ve směru osy z} - tento předpoklad platí pro planární problém.
\item {\bf Šíření vln v polární souřadnicové soustavě ve směru $\alpha$} - pro osově symetrický problém uvažujeme pouze tangenciální složku.
\end{itemize}
\newpage

\section{Metoda konečných prvků}
Základem samotné knihovny Hermes2D je numerické řešení pomocí tzv. metody konečných prvků.
%popis metody, trojúhelníky, ..

\section{Numerické řešení harmonické vlnové rovnice v kartézské souřadné soustavě}
Po zavedení zjednodušujících předpokladů vycházíme z rovnice ve tvaru
\begin{equation}
	\nabla^{2}\faz E_{(z)} +\faz k^{2}\faz E_{(z)} = 0
	\label{rce:sim_kar_helmholtz_num} 
\end{equation}
platné na definované oblasti $\Omega$, na které známe okrajové podmínky a ve které chceme dostat výsledné řešení. Tím může být například vnitřní prostor vlnovodu nebo rezonátoru. Nejprve rovnici (\ref{rce:sim_kar_helmholtz_num}) rozepíšeme na reálnou a imaginární složku
\begin{displaymath}
	\nabla^{2}(E_R + \mj E_I) + (\omega^{2}\varepsilon\mu - \mj\omega\mu\sigma)(E_R + \mj E_I) = 0,
\end{displaymath}
\begin{displaymath}
	\nabla^{2} E_R + \mj\nabla^{2} E_I + \omega^{2}\varepsilon\mu E_R + \mj\omega^{2}\varepsilon\mu E_I - \mj\omega\mu\sigma E_R + \omega\mu\sigma E_I = 0,
\end{displaymath}
kde reálnou část tvoří
\begin{equation}
	\Re : \nabla^{2} E_R + \omega^{2}\varepsilon\mu E_R + \omega\mu\sigma E_I = 0,
	\label{rce:sim_kar_num_real} 
\end{equation}
a imaginární je vyjádřena
\begin{equation}
	\Im : \nabla^{2} E_I + \omega^{2}\varepsilon\mu E_I - \omega\mu\sigma E_R = 0.
	\label{rce:sim_kar_num_imag} 
\end{equation}
Obě upravené rovnice (\ref{rce:sim_kar_num_real}) a (\ref{rce:sim_kar_num_imag}) převedeme do slabých forem, které splňují nulovou Dirichletovu a Neumannovu okrajovou podmínku. Postup převodu spočívá ve vynásobení parciálních diferenciálních rovnic testovací funkcí $v$ a v následné integraci přes oblast řešení $\Omega$ 
\begin{equation}
	\Re : \int_{\Omega}\nabla^{2} E_R\cdot v \dif S + \omega^{2}\varepsilon\mu\int_{\Omega} E_R\cdot v\dif S + \omega\mu\sigma\int_{\Omega} E_I\cdot v\dif S = 0,
	\label{rce:sim_kar_weak_odv_real} 
\end{equation}
\begin{equation}
	\Im : \int_{\Omega}\nabla^{2} E_I\cdot v\dif S + \omega^{2}\varepsilon\mu\int_{\Omega} E_I\cdot v\dif S - \omega\mu\sigma\int_{\Omega} E_R\cdot v\dif S = 0.
	\label{rce:sim_kar_weak_odv_imag} 
\end{equation}
V dalším kroku se aplikuje 1. Greenova identita \cite[příloha A.2]{num} (integrace po částech pro vyšší řády) a tím se získají slabé formy k původním rovnicím (\ref{rce:sim_kar_num_real}) a (\ref{rce:sim_kar_num_imag})
\begin{equation}
	\Re : \int_{\Gamma}\nabla E_R\cdot v\dif l-\int_{\Omega}\nabla E_R\cdot\nabla v \dif S + \omega^{2}\varepsilon\mu\int_{\Omega} E_R\cdot v\dif S + \omega\mu\sigma\int_{\Omega} E_I\cdot v\dif S = 0,
	\label{rce:sim_kar_weak_real} 
\end{equation}
\begin{equation}
	\Im : \int_{\Gamma}\nabla E_I\cdot v\dif l-\int_{\Omega}\nabla E_I\cdot\nabla v\dif S + \omega^{2}\varepsilon\mu\int_{\Omega} E_I\cdot v\dif S - \omega\mu\sigma\int_{\Omega} E_R\cdot v\dif S = 0.
	\label{rce:sim_kar_weak_imag} 
\end{equation}
První člen $\int_{\Gamma}\nabla E_R\cdot v\dif l$ respektive $\int_{\Gamma}\nabla E_I\cdot v\dif l$ vyjadřuje Neumanovu okrajovou podmínku. Pokud je podmínka nulová, tak i tyto členy budou nulové.



\subsection{Gaussova kvadratura}
\subsection{Řešení vlnové rovnice pomocí knihovny Hermes2D}

\section{Numerické řešení harmonické vlnové rovnice v polární souřadné soustavě}
Řešení vlnové rovnice v polárních souřadnicích vychází z rovnice
\begin{equation}
	\nabla\times(\nabla\times\vecfaz E) +\faz k^{2}\vecfaz E = 0.
	\label{rce:sim_pol_helmholtz_num} 
\end{equation}
Po zavedení zjednodušení, že výsledné řešení má pouze tangenciální složku, můžeme rovnici \ref{rce:sim_pol_helmholtz_num} snadno upravit. Nejprve vyjádříme vnitřní rotaci
\begin{displaymath}
	\nabla\times \frac{1}{r}\Bigg|
	\begin{array}{ccc}
\hat{r} & r\cdot\hat{\alpha} & \hat{z} \\
\frac{\partial}{\partial r} & \frac{\partial}{\partial \alpha} & \frac{\partial}{\partial z} \\
0 & r\cdot \faz E_{\alpha} & 0\\
\end{array}\Bigg| +\faz k^{2}\faz E_{\alpha} = 0,
\end{displaymath}
\begin{equation}
\nabla\times \Bigg[\hat{r}\bigg(-\frac{1}{r}\frac{\partial r\cdot\faz E_{\alpha}}{\partial z}\bigg) + r\cdot\hat{\alpha}\bigg(0\bigg) + \hat{z}\bigg(\frac{1}{r}\frac{\partial r\cdot\faz E_{\alpha}}{\partial r}\bigg) \Bigg]+\faz k^{2}\faz E_{\alpha} = 0.
	\label{rce:sim_pol_rotace1}
\end{equation}
Analogickým způsobem upravíme vnější rotaci ve vztahu (\ref{rce:sim_pol_rotace1})
\begin{displaymath}
	\frac{1}{r}\Bigg|
	\begin{array}{ccc}
\hat{r} & r\cdot\hat{\alpha} & \hat{z} \\
\frac{\partial}{\partial r} & \frac{\partial}{\partial \alpha} & \frac{\partial}{\partial z} \\
-\frac{1}{r}\frac{\partial r\cdot\faz E_{\alpha}}{\partial z} & 0 & \frac{1}{r}\frac{\partial r\cdot\faz E_{\alpha}}{\partial r}\\
\end{array}\Bigg| +\faz k^{2}\faz E_{\alpha} = 0,
\end{displaymath}
\begin{equation}
\Bigg[\hat{r}\bigg(\frac{1}{r^{2}}\frac{\partial^{2} r\cdot\faz E_{\alpha}}{\partial r\cdot\partial\alpha}\bigg) + r\cdot\hat{\alpha}\bigg(-\frac{1}{r^{2}}\frac{\partial^{2}r\cdot\faz E_{\alpha}}{\partial r^{2}} - \frac{1}{r}\frac{\partial^{2}r\cdot\faz E_{\alpha}}{\partial z^{2}}\bigg) + \hat{z}\bigg(0\bigg) \Bigg]+\faz k^{2}\faz E_{\alpha} = 0.
	\label{rce:sim_pol_rotace2}
\end{equation}
Ve výsledném vztahu (\ref{rce:sim_pol_rotace2}) zanedbáme všechny ostatní složky kromě té, která respektuje souřadnici $\hat{\alpha}$.



