% !TeX root = Main.tex
\section{Předpoklady simulace}
Vzhledem k široké problematice elektromagnetických vln je potřeba pro jejich modelování zavést některé zjednodušení. 
\begin{itemize}
\item {\bf Harmonická analýza} - první předpokladem je řešení vlnové rovnice v harmonickém tvaru (Helmholtzova rovnice)
\begin{displaymath}
	\nabla^{2}\vecfaz E +\faz k^{2}\vecfaz E = 0.
\end{displaymath}
\item {\bf Šíření vln v kartézské souřadnicové soustavě ve směru osy z} - touto úvahou předpokládáme složku ve směru osy $z = 0$, výsledek bude 2D charakteru.
\end{itemize}
\newpage

\section{Metoda konečných prvků}
%popis metody, trojúhelníky, ..

\section{Numerické řešení harmonické vlnové rovnice}
Po zavedení zjednodušujících předpokladů vycházíme z rovnice ve tvaru
\begin{equation}
	\nabla^{2}\vecfaz E_{(x,y)} +\faz k^{2}\vecfaz E_{(x,y)} = 0
	\label{rce:sim_helmholtz_num} 
\end{equation}
platné na definované oblasti $\Omega$, na které známe okrajové podmínky a ve které chceme dostat výsledné řešení. Tím může být například vnitřní prostor vlnovodu nebo rezonátoru. Nejprve z rovnice (\ref{rce:sim_helmholtz_num}) vyjádříme reálnou a imaginární složku
\begin{displaymath}
	\nabla^{2}(\vec E_R + \mj\vec E_I) + (\omega^{2}\varepsilon\mu - \mj\omega\mu\sigma)(\vec E_R + \mj\vec E_I) = 0,
\end{displaymath}
\begin{displaymath}
	\nabla^{2}\vec E_R + \mj\nabla^{2}\vec E_I + \omega^{2}\varepsilon\mu\vec E_R + \mj\omega^{2}\varepsilon\mu\vec E_I - \mj\omega\mu\sigma\vec E_R + \omega\mu\sigma\vec E_I = 0,
\end{displaymath}
kde reálnou část tvoří
\begin{equation}
	\Re : \nabla^{2}\vec E_R + \omega^{2}\varepsilon\mu\vec E_R + \mj\omega\mu\sigma\vec E_R = 0,
	\label{rce:sim_num_real} 
\end{equation}
a imaginární je vyjádřena
\begin{equation}
	\Im : \nabla^{2}\vec E_I + \omega^{2}\varepsilon\mu\vec E_I - \omega\mu\sigma\vec E_R = 0.
	\label{rce:sim_num_imag} 
\end{equation}
Obě upravené rovnice \ref{rce:sim_num_real} a \ref{rce:sim_num_imag} převedeme do slabých forem, které splňují nulovou Dirichletovu a Neumannovu okrajovou podmínku. Postup 


% úprava rovnice pro Ez, vytvoření slabé formy, numerické řešení integrálu - gauss kvadratura

\section{Řešení pomocí knihovny Hermes2D}
% real, imag.... + zápis v Qt



