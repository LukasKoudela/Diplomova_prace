% !TeX root = Main.tex
\section{Předpoklady simulace}
Vzhledem k široké problematice elektromagnetických vln je potřeba pro jejich modelování zavést některé zjednodušení. 
\begin{itemize}
\item {\bf Harmonická analýza} - první předpokladem je řešení vlnové rovnice v harmonickém tvaru (Helmholtzova rovnice)
\begin{displaymath}
	\nabla^{2}\vecfaz E +\faz k^{2}\vecfaz E = 0.
\end{displaymath}
\item {\bf Šíření vln v kartézské souřadnicové soustavě ve směru osy z} - touto úvahou předpokládáme složku ve směru osy $z = 0$, výsledek bude 2D charakteru.
\end{itemize}
\newpage

\section{Metoda konečných prvků}
%popis metody, trojúhelníky, ..

\section{Numerické řešení harmonické vlnové rovnice}
Po zavedení zjednodušujících předpokladů vycházíme z rovnice ve tvaru
\begin{equation}
	\nabla^{2}\faz E_{(z)} +\faz k^{2}\faz E_{(z)} = 0
	\label{rce:sim_helmholtz_num} 
\end{equation}
platné na definované oblasti $\Omega$, na které známe okrajové podmínky a ve které chceme dostat výsledné řešení. Tím může být například vnitřní prostor vlnovodu nebo rezonátoru. Nejprve rovnici (\ref{rce:sim_helmholtz_num}) rozepíšeme na reálnou a imaginární složku
\begin{displaymath}
	\nabla^{2}(E_R + \mj E_I) + (\omega^{2}\varepsilon\mu - \mj\omega\mu\sigma)(E_R + \mj E_I) = 0,
\end{displaymath}
\begin{displaymath}
	\nabla^{2} E_R + \mj\nabla^{2} E_I + \omega^{2}\varepsilon\mu E_R + \mj\omega^{2}\varepsilon\mu E_I - \mj\omega\mu\sigma E_R + \omega\mu\sigma E_I = 0,
\end{displaymath}
kde reálnou část tvoří
\begin{equation}
	\Re : \nabla^{2} E_R + \omega^{2}\varepsilon\mu E_R + \omega\mu\sigma E_I = 0,
	\label{rce:sim_num_real} 
\end{equation}
a imaginární je vyjádřena
\begin{equation}
	\Im : \nabla^{2} E_I + \omega^{2}\varepsilon\mu E_I - \omega\mu\sigma E_R = 0.
	\label{rce:sim_num_imag} 
\end{equation}
Obě upravené rovnice (\ref{rce:sim_num_real}) a (\ref{rce:sim_num_imag}) převedeme do slabých forem, které splňují nulovou Dirichletovu a Neumannovu okrajovou podmínku. Postup převodu spočívá ve vynásobení parciálních diferenciálních rovnic testovací funkcí $v$ a v následné integraci přes oblast řešení $\Omega$ 
\begin{equation}
	\Re : \int_{\Omega}\nabla^{2} E_R\cdot v \dif S + \omega^{2}\varepsilon\mu\int_{\Omega} E_R\cdot v\dif S + \omega\mu\sigma\int_{\Omega} E_I\cdot v\dif S = 0,
	\label{rce:sim_weak_odv_real} 
\end{equation}
\begin{equation}
	\Im : \int_{\Omega}\nabla^{2} E_I\cdot v\dif S + \omega^{2}\varepsilon\mu\int_{\Omega} E_I\cdot v\dif S - \omega\mu\sigma\int_{\Omega} E_R\cdot v\dif S = 0.
	\label{rce:sim_weak_odv_imag} 
\end{equation}
V dalším kroku se aplikuje 1. Greenova identita \cite[příloha A.2]{num} (integrace po částech pro vyšší řády) a tím se získají slabé formy k původním rovnicím (\ref{rce:sim_num_real}) a (\ref{rce:sim_num_imag})
\begin{equation}
	\Re : \int_{\Gamma}\nabla E_R\cdot v\dif l-\int_{\Omega}\nabla E_R\cdot\nabla v \dif S + \omega^{2}\varepsilon\mu\int_{\Omega} E_R\cdot v\dif S + \omega\mu\sigma\int_{\Omega} E_I\cdot v\dif S = 0,
	\label{rce:sim_weak_real} 
\end{equation}
\begin{equation}
	\Im : \int_{\Gamma}\nabla E_I\cdot v\dif l-\int_{\Omega}\nabla E_I\cdot\nabla v\dif S + \omega^{2}\varepsilon\mu\int_{\Omega} E_I\cdot v\dif S - \omega\mu\sigma\int_{\Omega} E_R\cdot v\dif S = 0.
	\label{rce:sim_weak_imag} 
\end{equation}
První člen $\int_{\Gamma}\nabla E_R\cdot v\dif l$ respektive $\int_{\Gamma}\nabla E_I\cdot v\dif l$ vyjadřuje Neumanovu okrajovou podmínku. Pokud je podmínka nulová, tak i tyto členy budou nulové.

\subsection{Gaussova kvadratura}

\section{Řešení vlnové rovnice pomocí knihovny Hermes2D}
% real, imag.... + zápis v Qt



