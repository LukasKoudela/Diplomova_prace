% !TeX root = Main.tex
Vznik tématu této diplomové práce souvisí s~vývojem aplikace Agros2d na Katedře teoretické elektrotechniky Fakulty elektrotechnické Západočeské univerzity v~Plzni. Tento univerzální multiplatformín program distribuovaný pod GNU GPL licencí slouží k~řešení fyzikálních polí. Umožňuje simulovat jevy v elektrostatickém, elektrickém proudovém, magnetickém nebo teplotním poli a dále řešit problémy týkající se termoelasticity. Poskytuje analýzu ustálených stavů a přechodných dějů.  Hlavním přínosem v~rámci diplomové práce bude s~využitím knihovny Hermes2D doplnění dalšího fyzikálního pole, které bude zobrazovat chování elektromagnetického pole, popsaného vlnovými rovnicemi. Práce si tedy klade za cíl především doplnit vyvíjenou aplikaci o další modul, čímž by rozšířila možnosti jejího použití do dalších technických oborů. 

Dalším přínosem bude nasimulovat vybrané vlnové problémy pro účely výuky předmětu KTE/EV. Jedná se o~příklady šíření elektromagnetických vln na dipólu, na rozhraní dvou prostředí, ve vrstveném prostředí, ve vlnovodných strukturách a v~dutinových rezonátorech.
