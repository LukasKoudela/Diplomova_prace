% !TeX root = Main.tex
Vznik tématu této diplomové práce souvisí s~vývojem aplikace Agros2d na Katedře teoretické elektrotechniky Fakulty elektrotechnické Západočeské univerzity v~Plzni. Tento univerzální multiplatformín program distribuovaný pod GNU GPL licencí slouží k~řešení fyzikálních polí. K tomu využívá knihovnu Hermes2D, jejíž základ spočívá v metodě konečných prvků vyšších řádů přesnosti.  

Umožňuje v současné době simulovat jevy v elektrostatickém, proudovém elektrickém, magnetickém nebo teplotním poli a dále řešit problémy týkající se termoelasticity a deformací. Podle typu fyzikálního pole poskytuje řešení ustálených stavů, přechodových dějů nebo harmonické analýzy.  

Hlavním přínosem v~rámci diplomové práce bude s~využitím knihovny Hermes2D doplnění dalšího fyzikálního pole, které bude zobrazovat chování elektromagnetického pole, popsaného vlnovými rovnicemi. Práce si tedy klade za cíl především doplnit vyvíjenou aplikaci o další modul, čímž by rozšířila možnosti jejího použití do dalších technických oborů. 
