% !TeX root = Main.tex
Vznik tématu této diplomové práce souvisí s~vývojem aplikace Agros2D \footnote{\texttt{http://agros2d.org/}} na Katedře teoretické elektrotechniky Fakulty elektrotechnické Západočeské univerzity v~Plzni. Tento univerzální multiplatformín program distribuovaný pod GNU GPL licencí slouží k~řešení fyzikálních polí. K~tomu využívá knihovnu Hermes2D\footnote{\texttt{http://hpfem.org/hermes2d}}, jejíž základ spočívá v~metodě konečných prvků vyšších řádů přesnosti ($hp$-FEM). Tato metoda využívá tzv.~$hp$-adaptivity, která umožňuje rychle dosáhnout přesných výsledků numerickým řešením.

Aplikace Agros 2D představuje v~současné době nástroj pro simulaci jevů v~elektrostatickém, proudovém elektrickém, magnetickém nebo teplotním poli. Dále je možné řešit problémy týkající se akustiky, termoelasticity a deformací. Podle typu fyzikálního pole poskytuje řešení ustálených stavů, přechodových dějů nebo harmonické analýzy.  

Hlavním přínosem v~rámci diplomové práce bude s~využitím knihovny Hermes2D doplnění dalšího fyzikálního pole, které bude zobrazovat chování elektromagnetického pole popsaného vlnovými rovnicemi. Práce si tedy klade za cíl především doplnit vyvíjenou aplikaci o~další modul, čímž by rozšířila možnosti jejího použití do dalších technických oborů. 
