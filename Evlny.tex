% !TeX root = Main.tex

\section{Šíření ve volném prostředí}
\subsection{Obecná vlnová rovnice}
Elektromagnetickým vlněním je označováno šíření nestacionárního elektromagnetického pole. Jedná se o~vektorové pole, pro jehož obecný popis je potřeba znát následující čtveřici vektorů pole v~každém bodě prostoru.\\ 

\begin{tabular}{lll}
$\vec E$ & $[V/m]$ & intenzita elektrického pole\\
$\vec D$ & $[C/m^{2}]$ & indukce elektrická\\
$\vec B$ & $[Wb/m^{2}],[T]$ & indukce magnetická\\
$\vec H$ & $[A/m]$ & intenzita magnetického pole\\
\end{tabular}\bigskip \\
Pro řešení pole ve vakuu nebo v~prostředí se známými materiálovými konstantami postačuje jakákoliv dvojice vektorů, kdy jeden je vektorem elektrického pole a druhý magentického pole. To znamená, že libovolná dvojice z~variant $\vec E\vec B$, $\vec E\vec H$, $\vec D\vec H$ a~$\vec D\vec B$ postačuje po popis a řešení elektromagnetického pole.
Relace mezi vektrory indukce a~intenzity popisují parametry prostředí. 
\begin{equation}
\vec D = \varepsilon\vec E = \varepsilon_{0}\varepsilon_{r}\vec E
\label{rce:D_epsE}
\end{equation}
\begin{equation}
\vec B = \mu\vec H = \mu_{0}\mu_{r}\vec H
\label{rce:B_muH}
\end{equation}
Pro úplnost je v~tabulce \ref{tab:evlny_vlastnosti_prostredi} přehled možných materiálových parametrů prostředí včetně hodnot pro vakuum, které jsou odlišené indexem $0$.
\begin{table}[!h]
\catcode`\-=12
\begin{center}
  	\caption{Materiálové vlastnosti prostředí}
  	\label{tab:evlny_vlastnosti_prostredi}
\begin{tabular}{l||llp{1cm}l}
	permitivita		& $\varepsilon$	& [F/m] & & $\varepsilon_{0} = 8,854.10^{-12} $\\
	\hline
	permeabilita	& $\mu$			& [H/m] & & $\mu_{0} = 4\pi.10^{-7} $\\
	\hline
	měrná vodivost	& $\sigma$		& [S/m] & &\\
\end{tabular}
\end{center}
\end{table}

Podle \cite[str.33]{emp} existují dva možné přístupy k~řešení polí, související s~časovou proměnností pole.
\begin{description}
\item[Stacionární pole]Určují se všechny zdroje a následně ke každému z~nich jejich prostorové rozložení pole. Výsledkem je superpozice všech polí.
\item[Nestacionární pole]Sestavíme diferenciální rovnici pro některý z~vektorů pole. Následně nalezneme obecné řešení a dle okrajových podmínek vybereme nejhodnější.
\end{description}
 
Podrobné odvození vlnových rovnic pro vektory pole $\vec E$ a $\vec B$, pro časově neměnné parametry prostředí, se nachází v~příloze \ref{kap:Odvozeni_VlnR}.
 
\subsection{Harmonické pole}

\subsection{Rovinná vlna}
str.40\\
\newpage

\section{Rozhraní dvou prostředí}
str.86, 91, 92, 94, 95, 96
\newpage

\section{Vrstvené prostředí}
str.103\\
str.104 - čtvrtvlnny transf\\
str.107 půlvlnné diel okno\\
\newpage

\section{Vlnovody}
\subsection{Klasifikace vln}
str.112 - TEM, TE, TM, HE, EH\\
str.108 - obecně\\
str.131 - TE, Ez = 0, TM Hz = 0\\
str.143 - Snell\\
\subsection{Kritická frekvence}
str.132
\newpage

\section{Dutinové rezonátory}
str.150\\
str.151 - rez. kmitočet\\
\newpage