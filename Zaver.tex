% !TeX root = Main.tex

Hlavním posláním této diplomové práce bylo vytvořit prostředek, pomocí kterého by bylo možné simulovat rozložení elektromagnetického pole při znalosti materiálových parametrů v řešené oblasti a okrajových podmínek. Pro numerické řešení harmonické vlnové rovnice byla využita knihova Hermes2D, založená na metodě konečných prvků vyšších řádů přesnosti.

Celý simulační kód byl koncipován jako modul pro multiplatformní aplikaci Agros2D.  


Možnosti simulačního modulu byly demonstrovány na příkladu řešení vlnovodu R100 v kapitole \ref{kap:Priklad}. Ověření výsledků bylo provedeno na stejném příkladu pomocí profesionálního programu COMSOL. Z porovnání výsledků je patrné, že rozložení elektromagnetického pole je u obou aplikací téměř identické. Drobné odlišnosti jsou způsobeny především přesností numerického řešení, která je závislá na míře diskretizace řešené oblasti ($h$-FEM) a řádu polynomu použitých prvků ($p$-FEM).

Příslušné parciální diferenciální rovnice byly odvozeny z Maxwellových rovnic.

